\documentclass{beamer}
\usetheme{CambridgeUS}
\usepackage{graphicx} % Required for inserting images
\graphicspath{./images}
\usecolortheme{dolphin}

% Define team members
\newcommand{\teamMember}[1]{\textbf{#1}}

\title{AirPic}
\author[Jayashre SaiSree Pranathi Navya Aanya]{\teamMember{Jayashre}\and \teamMember{SaiSree Kodali}\and \teamMember{Pranathi M}\and \teamMember{Navya Nayer}\and \teamMember{Aanya Chauhan}}
\date{\today}


\begin{document}

\begin{frame}
\titlepage
\end{frame}



\begin{frame}
\frametitle{Overview}
\begin{block}{To capture moments effortlessly with an intuitive gesture-powered photo experience that sets a new standard in camera interaction.}
\end{block}
\end{frame}

\begin{frame}
\frametitle{Tech Stack}
\begin{block}{App Development}
\begin{itemize}
    \item Android Studio - Local IDE
    \item Kotlin
    \item Android Splash Screen API
    \item Accompanist Permissions Library - Handling Audio and Camera Permissions
    \item Android ViewModel - Store \& Manage UI - related Data
    \item Android Jetpack Navigation Component (NavHost and NavController classes) - Navigation between fragments
    \item Android Jetpack Compose (Compose Modifier, LaunchedEffect) - UI Development
    \item Android CameraX API (CameraController, Preview View, LifecycleCameraController classes) - Camera Operations
\end{itemize}
\end{block}
\end{frame}

\begin{frame}
\frametitle{Tech Stack}
\begin{block}{App Development}
    \begin{itemize}
    \item Kotlin CoRoutines - Asynchronous Programming in Camera and Background Operations
    \item Android Material Design Component - Button, IconButton, BottomSheetScaffold, TopAppBar, Image, and Text
    \item Android MediaStore API - To store the captured images and videos in the device media's database
    \item Toast - Displays short-duration messages
    \item ContextCompat - Used to ensure that the app can run on different Android versions
    \end{itemize}
\end{block}
\end{frame}

\begin{frame}
\frametitle{Tech Stack}
\begin{block} {Smile Detection}
\begin{itemize}
    \item Google Colab - Online IDE
    \item Visual Studio Code - Local IDE
    \item Tensorflow's Keras API
    \item VGG16 (Visual Geometry Group 16) - a pre-trained feature extractor to capture image patterns, enhancing the model's ability to detect smiles.
\end{itemize}
    
\end{block}
\begin{block}{Hand Gesture Detection for Capturing Photos \& Zooming In and Out}
\begin{itemize}
    \item Visual Studio Code - Local IDE
    \item OpenCV (imutils) - capturing video from the camera
    \item Mediapipe (Hands) - Hand tracking model
    \item NumPy -  To handle numerical operations 
\end{itemize}
\end{block}
\end{frame}

\begin{frame}
\frametitle{Timeline}

\end{frame}

\begin{frame}
\frametitle{Timeline}

\end{frame}


\begin{frame}
\frametitle{Project Limitations}
\begin{block} {Challenges and Progress in Integrating Models with Camera App Development}
\end{block}
\end{frame}

%\begin{frame}
%\frametitle{Tech Stack}
%\begin{block}{AR Filters and Colour Filters}
%MediaPipe - Face detection\\
%OpenCV's Facemark API - Face features detection \\
%Makesense - Labelling the feature points in the photo \\
%Photoshop, free to use images - Filters Deploying\\
%Lookup Table Library - Colour Filters\\
%\end{block}
%\end{frame}



%%
%%\begin{frame}
%%\frametitle{Tech Stack}
%%\begin{block}{Integration}
%%Pyjnius - Python - to - Java Bridge\\
%%Setting up Python Environment\\
%%Preparing Python-to-Kotlin integration.\\
%%Python code using Bee Ware Library for Android execution.\\
%%\end{block}
%%\end{frame}

\begin{frame}
\frametitle{Demo}
\begin{block}{We will be demonstrating the following models}
\begin{itemize}
    \item AirPic: The App
    \item Smile Detection Model
    \item Palm Detection Model
    \item Gesture Model for Zooming In and Out
\end{itemize}
\end{block}
\end{frame}

\begin{frame}
\frametitle{Learnings}
\end{frame}

\begin{frame}
\frametitle{Future Scope}
\begin{block} {1. To implement customized in-app settings such as grids and HDR.}
\end{block}
\begin{block} {2. Create a centralized in-app Gallery for efficient management and viewing of images and videos, enhancing user experience}
\end{block}
\begin{block}{3. Develop our project into an iOS App}
\end{block}
\begin{block}{4. Enrich user experience by incorporating intuitive gestures like turning on the video or activating the timer.}
\end{block}
\begin{block}{5. Transform our project into a vibrant social platform, fostering connections and collaboration.}
\end{block}
\end{frame}

\begin{frame}
\frametitle{Conclusion \& Thank You}
\begin{block}
    {We value and appreciate your feedback.}
\end{block}
\end{frame}

\end{document}

